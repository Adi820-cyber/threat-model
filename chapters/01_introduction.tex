\section*{Introduction}
\subsection*{Why Threat Modeling?}
In the digital era, organizations face a rapidly evolving threat landscape. Cyberattacks are no longer limited to large corporations; small businesses, governments, and individuals are all targets. The consequences of a successful attack can be devastating: data breaches, financial loss, reputational damage, and regulatory penalties. Proactive security is essential, and threat modeling is a foundational practice for building secure systems.

\subsection*{What is Threat Modeling?}
Threat modeling is a structured process for identifying, evaluating, and addressing security threats. It enables defenders to "think like an attacker" and anticipate how systems might be compromised. By mapping out assets, entry points, and potential attack vectors, organizations can prioritize security controls and reduce risk before vulnerabilities are exploited.

\subsection*{Benefits of Threat Modeling}
\begin{itemize}
	\item \textbf{Proactive Defense:} Identify and mitigate risks early in the development lifecycle.
	\item \textbf{Cost Savings:} Addressing security issues during design is far less expensive than post-breach remediation.
	\item \textbf{Regulatory Compliance:} Many standards (e.g., GDPR, HIPAA, PCI DSS) require risk assessment and threat modeling.
	\item \textbf{Improved Communication:} Provides a common language for developers, security teams, and business stakeholders.
	\item \textbf{Continuous Improvement:} Threat models can be updated as systems evolve, supporting ongoing security.
\end{itemize}

\subsection*{Threat Modeling in the Secure Development Lifecycle}
Threat modeling is most effective when integrated into the Secure Development Lifecycle (SDL). It should be performed at the design stage, revisited during implementation, and updated as new features or threats emerge. Leading organizations, including Microsoft and Google, have made threat modeling a mandatory part of their SDL processes.

\subsection*{Who Should Perform Threat Modeling?}
Threat modeling is not just for security experts. Developers, architects, product managers, and even business analysts can contribute valuable insights. Cross-functional collaboration ensures that all aspects of the system are considered, from technical vulnerabilities to business logic flaws.

\subsection*{Chapter Overview}
This document provides a comprehensive guide to threat modeling, covering key frameworks (STRIDE, PASTA, Trike, VAST, OCTAVE, OWASP), practical methodologies, a real-world case study, and hands-on labs. Whether you are new to cybersecurity or an experienced practitioner, this report will help you build a robust, proactive defense against modern threats.
